% Generated by Sphinx.
\def\sphinxdocclass{report}
\documentclass[letterpaper,10pt,english]{sphinxmanual}
\usepackage[utf8]{inputenc}
\DeclareUnicodeCharacter{00A0}{\nobreakspace}
\usepackage[T1]{fontenc}
\usepackage{babel}
\usepackage{times}
\usepackage[Bjarne]{fncychap}
\usepackage{longtable}
\usepackage{sphinx}
\usepackage{multirow}


\title{qikify Documentation}
\date{March 29, 2012}
\release{}
\author{Author}
\newcommand{\sphinxlogo}{}
\renewcommand{\releasename}{Release}
\makeindex

\makeatletter
\def\PYG@reset{\let\PYG@it=\relax \let\PYG@bf=\relax%
    \let\PYG@ul=\relax \let\PYG@tc=\relax%
    \let\PYG@bc=\relax \let\PYG@ff=\relax}
\def\PYG@tok#1{\csname PYG@tok@#1\endcsname}
\def\PYG@toks#1+{\ifx\relax#1\empty\else%
    \PYG@tok{#1}\expandafter\PYG@toks\fi}
\def\PYG@do#1{\PYG@bc{\PYG@tc{\PYG@ul{%
    \PYG@it{\PYG@bf{\PYG@ff{#1}}}}}}}
\def\PYG#1#2{\PYG@reset\PYG@toks#1+\relax+\PYG@do{#2}}

\expandafter\def\csname PYG@tok@gd\endcsname{\def\PYG@tc##1{\textcolor[rgb]{0.63,0.00,0.00}{##1}}}
\expandafter\def\csname PYG@tok@gu\endcsname{\let\PYG@bf=\textbf\def\PYG@tc##1{\textcolor[rgb]{0.50,0.00,0.50}{##1}}}
\expandafter\def\csname PYG@tok@gt\endcsname{\def\PYG@tc##1{\textcolor[rgb]{0.00,0.25,0.82}{##1}}}
\expandafter\def\csname PYG@tok@gs\endcsname{\let\PYG@bf=\textbf}
\expandafter\def\csname PYG@tok@gr\endcsname{\def\PYG@tc##1{\textcolor[rgb]{1.00,0.00,0.00}{##1}}}
\expandafter\def\csname PYG@tok@cm\endcsname{\let\PYG@it=\textit\def\PYG@tc##1{\textcolor[rgb]{0.25,0.50,0.56}{##1}}}
\expandafter\def\csname PYG@tok@vg\endcsname{\def\PYG@tc##1{\textcolor[rgb]{0.73,0.38,0.84}{##1}}}
\expandafter\def\csname PYG@tok@m\endcsname{\def\PYG@tc##1{\textcolor[rgb]{0.13,0.50,0.31}{##1}}}
\expandafter\def\csname PYG@tok@mh\endcsname{\def\PYG@tc##1{\textcolor[rgb]{0.13,0.50,0.31}{##1}}}
\expandafter\def\csname PYG@tok@cs\endcsname{\def\PYG@tc##1{\textcolor[rgb]{0.25,0.50,0.56}{##1}}\def\PYG@bc##1{\setlength{\fboxsep}{0pt}\colorbox[rgb]{1.00,0.94,0.94}{\strut ##1}}}
\expandafter\def\csname PYG@tok@ge\endcsname{\let\PYG@it=\textit}
\expandafter\def\csname PYG@tok@vc\endcsname{\def\PYG@tc##1{\textcolor[rgb]{0.73,0.38,0.84}{##1}}}
\expandafter\def\csname PYG@tok@il\endcsname{\def\PYG@tc##1{\textcolor[rgb]{0.13,0.50,0.31}{##1}}}
\expandafter\def\csname PYG@tok@go\endcsname{\def\PYG@tc##1{\textcolor[rgb]{0.19,0.19,0.19}{##1}}}
\expandafter\def\csname PYG@tok@cp\endcsname{\def\PYG@tc##1{\textcolor[rgb]{0.00,0.44,0.13}{##1}}}
\expandafter\def\csname PYG@tok@gi\endcsname{\def\PYG@tc##1{\textcolor[rgb]{0.00,0.63,0.00}{##1}}}
\expandafter\def\csname PYG@tok@gh\endcsname{\let\PYG@bf=\textbf\def\PYG@tc##1{\textcolor[rgb]{0.00,0.00,0.50}{##1}}}
\expandafter\def\csname PYG@tok@ni\endcsname{\let\PYG@bf=\textbf\def\PYG@tc##1{\textcolor[rgb]{0.84,0.33,0.22}{##1}}}
\expandafter\def\csname PYG@tok@nl\endcsname{\let\PYG@bf=\textbf\def\PYG@tc##1{\textcolor[rgb]{0.00,0.13,0.44}{##1}}}
\expandafter\def\csname PYG@tok@nn\endcsname{\let\PYG@bf=\textbf\def\PYG@tc##1{\textcolor[rgb]{0.05,0.52,0.71}{##1}}}
\expandafter\def\csname PYG@tok@no\endcsname{\def\PYG@tc##1{\textcolor[rgb]{0.38,0.68,0.84}{##1}}}
\expandafter\def\csname PYG@tok@na\endcsname{\def\PYG@tc##1{\textcolor[rgb]{0.25,0.44,0.63}{##1}}}
\expandafter\def\csname PYG@tok@nb\endcsname{\def\PYG@tc##1{\textcolor[rgb]{0.00,0.44,0.13}{##1}}}
\expandafter\def\csname PYG@tok@nc\endcsname{\let\PYG@bf=\textbf\def\PYG@tc##1{\textcolor[rgb]{0.05,0.52,0.71}{##1}}}
\expandafter\def\csname PYG@tok@nd\endcsname{\let\PYG@bf=\textbf\def\PYG@tc##1{\textcolor[rgb]{0.33,0.33,0.33}{##1}}}
\expandafter\def\csname PYG@tok@ne\endcsname{\def\PYG@tc##1{\textcolor[rgb]{0.00,0.44,0.13}{##1}}}
\expandafter\def\csname PYG@tok@nf\endcsname{\def\PYG@tc##1{\textcolor[rgb]{0.02,0.16,0.49}{##1}}}
\expandafter\def\csname PYG@tok@si\endcsname{\let\PYG@it=\textit\def\PYG@tc##1{\textcolor[rgb]{0.44,0.63,0.82}{##1}}}
\expandafter\def\csname PYG@tok@s2\endcsname{\def\PYG@tc##1{\textcolor[rgb]{0.25,0.44,0.63}{##1}}}
\expandafter\def\csname PYG@tok@vi\endcsname{\def\PYG@tc##1{\textcolor[rgb]{0.73,0.38,0.84}{##1}}}
\expandafter\def\csname PYG@tok@nt\endcsname{\let\PYG@bf=\textbf\def\PYG@tc##1{\textcolor[rgb]{0.02,0.16,0.45}{##1}}}
\expandafter\def\csname PYG@tok@nv\endcsname{\def\PYG@tc##1{\textcolor[rgb]{0.73,0.38,0.84}{##1}}}
\expandafter\def\csname PYG@tok@s1\endcsname{\def\PYG@tc##1{\textcolor[rgb]{0.25,0.44,0.63}{##1}}}
\expandafter\def\csname PYG@tok@gp\endcsname{\let\PYG@bf=\textbf\def\PYG@tc##1{\textcolor[rgb]{0.78,0.36,0.04}{##1}}}
\expandafter\def\csname PYG@tok@sh\endcsname{\def\PYG@tc##1{\textcolor[rgb]{0.25,0.44,0.63}{##1}}}
\expandafter\def\csname PYG@tok@ow\endcsname{\let\PYG@bf=\textbf\def\PYG@tc##1{\textcolor[rgb]{0.00,0.44,0.13}{##1}}}
\expandafter\def\csname PYG@tok@sx\endcsname{\def\PYG@tc##1{\textcolor[rgb]{0.78,0.36,0.04}{##1}}}
\expandafter\def\csname PYG@tok@bp\endcsname{\def\PYG@tc##1{\textcolor[rgb]{0.00,0.44,0.13}{##1}}}
\expandafter\def\csname PYG@tok@c1\endcsname{\let\PYG@it=\textit\def\PYG@tc##1{\textcolor[rgb]{0.25,0.50,0.56}{##1}}}
\expandafter\def\csname PYG@tok@kc\endcsname{\let\PYG@bf=\textbf\def\PYG@tc##1{\textcolor[rgb]{0.00,0.44,0.13}{##1}}}
\expandafter\def\csname PYG@tok@c\endcsname{\let\PYG@it=\textit\def\PYG@tc##1{\textcolor[rgb]{0.25,0.50,0.56}{##1}}}
\expandafter\def\csname PYG@tok@mf\endcsname{\def\PYG@tc##1{\textcolor[rgb]{0.13,0.50,0.31}{##1}}}
\expandafter\def\csname PYG@tok@err\endcsname{\def\PYG@bc##1{\setlength{\fboxsep}{0pt}\fcolorbox[rgb]{1.00,0.00,0.00}{1,1,1}{\strut ##1}}}
\expandafter\def\csname PYG@tok@kd\endcsname{\let\PYG@bf=\textbf\def\PYG@tc##1{\textcolor[rgb]{0.00,0.44,0.13}{##1}}}
\expandafter\def\csname PYG@tok@ss\endcsname{\def\PYG@tc##1{\textcolor[rgb]{0.32,0.47,0.09}{##1}}}
\expandafter\def\csname PYG@tok@sr\endcsname{\def\PYG@tc##1{\textcolor[rgb]{0.14,0.33,0.53}{##1}}}
\expandafter\def\csname PYG@tok@mo\endcsname{\def\PYG@tc##1{\textcolor[rgb]{0.13,0.50,0.31}{##1}}}
\expandafter\def\csname PYG@tok@mi\endcsname{\def\PYG@tc##1{\textcolor[rgb]{0.13,0.50,0.31}{##1}}}
\expandafter\def\csname PYG@tok@kn\endcsname{\let\PYG@bf=\textbf\def\PYG@tc##1{\textcolor[rgb]{0.00,0.44,0.13}{##1}}}
\expandafter\def\csname PYG@tok@o\endcsname{\def\PYG@tc##1{\textcolor[rgb]{0.40,0.40,0.40}{##1}}}
\expandafter\def\csname PYG@tok@kr\endcsname{\let\PYG@bf=\textbf\def\PYG@tc##1{\textcolor[rgb]{0.00,0.44,0.13}{##1}}}
\expandafter\def\csname PYG@tok@s\endcsname{\def\PYG@tc##1{\textcolor[rgb]{0.25,0.44,0.63}{##1}}}
\expandafter\def\csname PYG@tok@kp\endcsname{\def\PYG@tc##1{\textcolor[rgb]{0.00,0.44,0.13}{##1}}}
\expandafter\def\csname PYG@tok@w\endcsname{\def\PYG@tc##1{\textcolor[rgb]{0.73,0.73,0.73}{##1}}}
\expandafter\def\csname PYG@tok@kt\endcsname{\def\PYG@tc##1{\textcolor[rgb]{0.56,0.13,0.00}{##1}}}
\expandafter\def\csname PYG@tok@sc\endcsname{\def\PYG@tc##1{\textcolor[rgb]{0.25,0.44,0.63}{##1}}}
\expandafter\def\csname PYG@tok@sb\endcsname{\def\PYG@tc##1{\textcolor[rgb]{0.25,0.44,0.63}{##1}}}
\expandafter\def\csname PYG@tok@k\endcsname{\let\PYG@bf=\textbf\def\PYG@tc##1{\textcolor[rgb]{0.00,0.44,0.13}{##1}}}
\expandafter\def\csname PYG@tok@se\endcsname{\let\PYG@bf=\textbf\def\PYG@tc##1{\textcolor[rgb]{0.25,0.44,0.63}{##1}}}
\expandafter\def\csname PYG@tok@sd\endcsname{\let\PYG@it=\textit\def\PYG@tc##1{\textcolor[rgb]{0.25,0.44,0.63}{##1}}}

\def\PYGZbs{\char`\\}
\def\PYGZus{\char`\_}
\def\PYGZob{\char`\{}
\def\PYGZcb{\char`\}}
\def\PYGZca{\char`\^}
\def\PYGZam{\char`\&}
\def\PYGZlt{\char`\<}
\def\PYGZgt{\char`\>}
\def\PYGZsh{\char`\#}
\def\PYGZpc{\char`\%}
\def\PYGZdl{\char`\$}
\def\PYGZti{\char`\~}
% for compatibility with earlier versions
\def\PYGZat{@}
\def\PYGZlb{[}
\def\PYGZrb{]}
\makeatother

\begin{document}

\maketitle
\tableofcontents
\phantomsection\label{index::doc}


Contents:


\chapter{qikify Package}
\label{qikify:qikify-package}\label{qikify:welcome-to-qikify-s-documentation}\label{qikify::doc}

\section{\texttt{qikify} Package}
\label{qikify:id1}\phantomsection\label{qikify:module-qikify.__init__}\index{qikify.\_\_init\_\_ (module)}

\section{\texttt{helpers} Module}
\label{qikify:module-qikify.helpers}\label{qikify:helpers-module}\index{qikify.helpers (module)}\index{bool2symmetric() (in module qikify.helpers)}

\begin{fulllineitems}
\phantomsection\label{qikify:qikify.helpers.bool2symmetric}\pysiglinewithargsret{\code{qikify.helpers.}\bfcode{bool2symmetric}}{\emph{data}}{}
Changes True/False data to +1/-1 symmetric.

\end{fulllineitems}

\index{computeR2() (in module qikify.helpers)}

\begin{fulllineitems}
\phantomsection\label{qikify:qikify.helpers.computeR2}\pysiglinewithargsret{\code{qikify.helpers.}\bfcode{computeR2}}{\emph{yhat}, \emph{y}}{}
Computes R-squared coefficient of determination.
\begin{quote}

R2 = 1 - sum((y\_hat - y\_test)**2) / sum((y\_test - np.mean(y\_test))**2)
\end{quote}
\begin{quote}\begin{description}
\item[{Parameters }] \leavevmode
\textbf{yhat} : 1d array or list of floats -- estimated values of y

\textbf{y} : 1d array or list of floats -- true values

\end{description}\end{quote}
\paragraph{Examples}

r2 = computeR2(yhat, y)

\end{fulllineitems}

\index{create\_logger() (in module qikify.helpers)}

\begin{fulllineitems}
\phantomsection\label{qikify:qikify.helpers.create_logger}\pysiglinewithargsret{\code{qikify.helpers.}\bfcode{create\_logger}}{\emph{logmodule}}{}
\end{fulllineitems}

\index{getParetoFront() (in module qikify.helpers)}

\begin{fulllineitems}
\phantomsection\label{qikify:qikify.helpers.getParetoFront}\pysiglinewithargsret{\code{qikify.helpers.}\bfcode{getParetoFront}}{\emph{data}}{}
Extracts the 2D Pareto-optimal front from a 2D numpy array.
\begin{quote}\begin{description}
\item[{Parameters }] \leavevmode
\textbf{data} : numpy ndarray, or pandas.DataFrame
\begin{quote}

Data for which we want pareto-optimal front.
\end{quote}

\end{description}\end{quote}
\paragraph{Examples}

p = getParetoFront(data)

\end{fulllineitems}

\index{is1D() (in module qikify.helpers)}

\begin{fulllineitems}
\phantomsection\label{qikify:qikify.helpers.is1D}\pysiglinewithargsret{\code{qikify.helpers.}\bfcode{is1D}}{\emph{data}}{}
Determine if data is 1-dimensional.

\end{fulllineitems}

\index{nmse() (in module qikify.helpers)}

\begin{fulllineitems}
\phantomsection\label{qikify:qikify.helpers.nmse}\pysiglinewithargsret{\code{qikify.helpers.}\bfcode{nmse}}{\emph{yhat}, \emph{y}, \emph{min\_y=None}, \emph{max\_y=None}}{}
Calculates the normalized mean-squared error.
\begin{quote}\begin{description}
\item[{Parameters }] \leavevmode
\textbf{yhat} : 1d array or list of floats
\begin{quote}

estimated values of y
\end{quote}

\textbf{y} : 1d array or list of floats
\begin{quote}

true values
\end{quote}

\textbf{min\_y, max\_y} : float, float
\begin{quote}

roughly the min and max; they do not have to be the perfect values of min and max, because
they're just here to scale the output into a roughly {[}0,1{]} range
\end{quote}

\end{description}\end{quote}
\paragraph{Examples}

nmse = nmse(yhat, y)

\end{fulllineitems}

\index{partition() (in module qikify.helpers)}

\begin{fulllineitems}
\phantomsection\label{qikify:qikify.helpers.partition}\pysiglinewithargsret{\code{qikify.helpers.}\bfcode{partition}}{\emph{data}, \emph{threshold=0.5}, \emph{verbose=False}}{}
Partitions data into training and test sets. Assumes the last column of
data is y.
\begin{quote}\begin{description}
\item[{Parameters }] \leavevmode
\textbf{data} : numpy ndarray, or pandas.DataFrame
\begin{quote}

Data to partition into training and test sets.
\end{quote}

\textbf{threshold} : float
\begin{quote}

Determines ratio of training : test.
\end{quote}

\end{description}\end{quote}
\paragraph{Examples}

TODO

\end{fulllineitems}

\index{standardize() (in module qikify.helpers)}

\begin{fulllineitems}
\phantomsection\label{qikify:qikify.helpers.standardize}\pysiglinewithargsret{\code{qikify.helpers.}\bfcode{standardize}}{\emph{X}, \emph{scaleDict=None}, \emph{reverse=False}}{}
Facilitates standardizing data by subtracting the mean and dividing by
the standard deviation. Set reverse to True to perform the inverse 
operation.
\begin{quote}\begin{description}
\item[{Parameters }] \leavevmode
\textbf{X} : numpy ndarray, or pandas.DataFrame
\begin{quote}

Data for which we want pareto-optimal front.
\end{quote}

\textbf{scaleDict: dict, default None} :
\begin{quote}

Dictionary with elements mean/std to control standardization.
\end{quote}

\textbf{reverse: boolean, default False} :
\begin{quote}

If this flag is set, the standardization will be reversed; e.g.,
we take a dataset with zero mean and unit variance and change to
dataset with mean=scaleDict.mean and std=scaleDict.std.
\end{quote}

\end{description}\end{quote}
\paragraph{Examples}

TODO

\end{fulllineitems}

\index{zeroMatrixDiagonal() (in module qikify.helpers)}

\begin{fulllineitems}
\phantomsection\label{qikify:qikify.helpers.zeroMatrixDiagonal}\pysiglinewithargsret{\code{qikify.helpers.}\bfcode{zeroMatrixDiagonal}}{\emph{X}}{}
Set the diagonal of a matrix to all zeros.
\begin{quote}\begin{description}
\item[{Parameters }] \leavevmode
\textbf{X} : numpy ndarray
\begin{quote}

Matrix on which to zero out the diagonal.
\end{quote}

\end{description}\end{quote}
\paragraph{Examples}

Xp = zeroMatrixDiagonal(X)

\end{fulllineitems}



\section{\texttt{term\_helpers} Module}
\label{qikify:term-helpers-module}\label{qikify:module-qikify.term_helpers}\index{qikify.term\_helpers (module)}\index{colors (class in qikify.term\_helpers)}

\begin{fulllineitems}
\phantomsection\label{qikify:qikify.term_helpers.colors}\pysigline{\strong{class }\code{qikify.term\_helpers.}\bfcode{colors}}
Bases: \code{object}
\paragraph{Methods}

\begin{longtable}{ll}
\hline
\endfirsthead

\multicolumn{2}{c}%
{{\bfseries \tablename\ \thetable{} -- continued from previous page}} \\
\hline
\endhead

\hline \multicolumn{2}{|r|}{{Continued on next page}} \\ \hline
\endfoot

\hline
\endlastfoot


{\hyperref[qikify:qikify.term_helpers.colors.disable]{\code{disable}}}()
 & 

\\\hline
\end{longtable}

\index{disable() (qikify.term\_helpers.colors method)}

\begin{fulllineitems}
\phantomsection\label{qikify:qikify.term_helpers.colors.disable}\pysiglinewithargsret{\bfcode{disable}}{}{}
\end{fulllineitems}


\end{fulllineitems}

\index{outputPassFail() (in module qikify.term\_helpers)}

\begin{fulllineitems}
\phantomsection\label{qikify:qikify.term_helpers.outputPassFail}\pysiglinewithargsret{\code{qikify.term\_helpers.}\bfcode{outputPassFail}}{\emph{gnd}}{}
\end{fulllineitems}



\section{Subpackages}
\label{qikify:subpackages}

\subsection{controllers Package}
\label{qikify.controllers::doc}\label{qikify.controllers:controllers-package}

\subsubsection{\texttt{KDE} Module}
\label{qikify.controllers:module-qikify.controllers.KDE}\label{qikify.controllers:kde-module}\index{qikify.controllers.KDE (module)}\index{KDE (class in qikify.controllers.KDE)}

\begin{fulllineitems}
\phantomsection\label{qikify.controllers:qikify.controllers.KDE.KDE}\pysigline{\strong{class }\code{qikify.controllers.KDE.}\bfcode{KDE}}
Bases: \code{object}

This class implements non-parametric kernel density estimation.
\paragraph{Methods}

\begin{longtable}{ll}
\hline
\endfirsthead

\multicolumn{2}{c}%
{{\bfseries \tablename\ \thetable{} -- continued from previous page}} \\
\hline
\endhead

\hline \multicolumn{2}{|r|}{{Continued on next page}} \\ \hline
\endfoot

\hline
\endlastfoot


{\hyperref[qikify.controllers:qikify.controllers.KDE.KDE.run]{\code{run}}}(X{[}, specs, nSamples, counts, a, bounds{]})
 & 
Primary execution point.
\\\hline
\end{longtable}

\index{run() (qikify.controllers.KDE.KDE method)}

\begin{fulllineitems}
\phantomsection\label{qikify.controllers:qikify.controllers.KDE.KDE.run}\pysiglinewithargsret{\bfcode{run}}{\emph{X}, \emph{specs=None}, \emph{nSamples=0}, \emph{counts=None}, \emph{a=0}, \emph{bounds=None}}{}
Primary execution point. Run either standard KDE or class-membership based KDE. If 
any of the class-membership based KDE arguments are set, it will be run instead of 
standard KDE.
\begin{quote}\begin{description}
\item[{Parameters }] \leavevmode
\textbf{X} : array\_like
\begin{quote}

Contains data stored in a pandas.DataFrame.
\end{quote}

\textbf{nSamples} : int
\begin{quote}

The number of samples to generate.
\end{quote}

\textbf{specs} : qikify.models.Specs, optional
\begin{quote}

If using partitioned sampling, boundaries defining pass/critical/fail subspaces must be provided.
\end{quote}

\textbf{counts} : dict, optional
\begin{quote}

If using partitioned sampling, counts dictionary must be provided, with three keys: nGood, nCritical, nFail.
\end{quote}

\end{description}\end{quote}

\end{fulllineitems}


\end{fulllineitems}



\subsubsection{\texttt{LSFS} Module}
\label{qikify.controllers:lsfs-module}\label{qikify.controllers:module-qikify.controllers.LSFS}\index{qikify.controllers.LSFS (module)}\index{LSFS (class in qikify.controllers.LSFS)}

\begin{fulllineitems}
\phantomsection\label{qikify.controllers:qikify.controllers.LSFS.LSFS}\pysigline{\strong{class }\code{qikify.controllers.LSFS.}\bfcode{LSFS}}
Bases: \code{object}
\paragraph{Methods}

\begin{longtable}{ll}
\hline
\endfirsthead

\multicolumn{2}{c}%
{{\bfseries \tablename\ \thetable{} -- continued from previous page}} \\
\hline
\endhead

\hline \multicolumn{2}{|r|}{{Continued on next page}} \\ \hline
\endfoot

\hline
\endlastfoot


{\hyperref[qikify.controllers:qikify.controllers.LSFS.LSFS.constructS]{\code{constructS}}}(X, gnd{[}, k, t, bLDA, bSelfConnected{]})
 & 

\\\hline

{\hyperref[qikify.controllers:qikify.controllers.LSFS.LSFS.run]{\code{run}}}(Xin, gnd)
 & 
Run Laplacian Score Feature Selection.
\\\hline

{\hyperref[qikify.controllers:qikify.controllers.LSFS.LSFS.threshold]{\code{threshold}}}(T\_L)
 & 

\\\hline
\end{longtable}

\index{constructS() (qikify.controllers.LSFS.LSFS method)}

\begin{fulllineitems}
\phantomsection\label{qikify.controllers:qikify.controllers.LSFS.LSFS.constructS}\pysiglinewithargsret{\bfcode{constructS}}{\emph{X}, \emph{gnd}, \emph{k=0}, \emph{t=1}, \emph{bLDA=False}, \emph{bSelfConnected=True}}{}
\end{fulllineitems}

\index{run() (qikify.controllers.LSFS.LSFS method)}

\begin{fulllineitems}
\phantomsection\label{qikify.controllers:qikify.controllers.LSFS.LSFS.run}\pysiglinewithargsret{\bfcode{run}}{\emph{Xin}, \emph{gnd}}{}
Run Laplacian Score Feature Selection.

\begin{notice}{note}{Note:}
Eventually, it'd be nice to maintain col names with Xin so that we can add a plot method to plot scores vs. column names.
\end{notice}
\begin{quote}\begin{description}
\item[{Parameters }] \leavevmode
\textbf{Xin} : array\_like
\begin{quote}

A numpy.ndarray or pandas.DataFrame, with rows corresponding to observations and columns to features.
\end{quote}

\textbf{gnd} : array\_like
\begin{quote}

A numpy.ndarray or pandas.DataFrame pass/fail vector of the same dimension as Xin
\end{quote}

\end{description}\end{quote}
\paragraph{Notes}

This code is based on the definition from the paper {\hyperref[qikify.controllers:r1]{{[}R1{]}}}:

\end{fulllineitems}

\index{threshold() (qikify.controllers.LSFS.LSFS method)}

\begin{fulllineitems}
\phantomsection\label{qikify.controllers:qikify.controllers.LSFS.LSFS.threshold}\pysiglinewithargsret{\bfcode{threshold}}{\emph{T\_L}}{}
\end{fulllineitems}


\end{fulllineitems}



\subsubsection{\texttt{OLS} Module}
\label{qikify.controllers:module-qikify.controllers.OLS}\label{qikify.controllers:ols-module}\index{qikify.controllers.OLS (module)}\index{OLS (class in qikify.controllers.OLS)}

\begin{fulllineitems}
\phantomsection\label{qikify.controllers:qikify.controllers.OLS.OLS}\pysigline{\strong{class }\code{qikify.controllers.OLS.}\bfcode{OLS}}
Bases: \code{object}

Ordinary least squares multivariate regression.
\paragraph{Methods}

\begin{longtable}{ll}
\hline
\endfirsthead

\multicolumn{2}{c}%
{{\bfseries \tablename\ \thetable{} -- continued from previous page}} \\
\hline
\endhead

\hline \multicolumn{2}{|r|}{{Continued on next page}} \\ \hline
\endfoot

\hline
\endlastfoot


{\hyperref[qikify.controllers:qikify.controllers.OLS.OLS.JB]{\code{JB}}}()
 & 
Calculate residual skewness, kurtosis, and do the JB test for normality
\\\hline

{\hyperref[qikify.controllers:qikify.controllers.OLS.OLS.computeStatistics]{\code{computeStatistics}}}()
 & 

\\\hline

{\hyperref[qikify.controllers:qikify.controllers.OLS.OLS.dw]{\code{dw}}}()
 & 
Calculates the Durbin-Waston statistic
\\\hline

{\hyperref[qikify.controllers:qikify.controllers.OLS.OLS.ll]{\code{ll}}}()
 & 
Calculate model log-likelihood and two information criteria
\\\hline

{\hyperref[qikify.controllers:qikify.controllers.OLS.OLS.omni]{\code{omni}}}()
 & 
Omnibus test for normality
\\\hline

{\hyperref[qikify.controllers:qikify.controllers.OLS.OLS.train]{\code{train}}}(X, y{[}, useQR, addConstant{]})
 & 
Solve y = Xb.
\\\hline
\end{longtable}

\index{JB() (qikify.controllers.OLS.OLS method)}

\begin{fulllineitems}
\phantomsection\label{qikify.controllers:qikify.controllers.OLS.OLS.JB}\pysiglinewithargsret{\bfcode{JB}}{}{}
Calculate residual skewness, kurtosis, and do the JB test for normality

\end{fulllineitems}

\index{computeStatistics() (qikify.controllers.OLS.OLS method)}

\begin{fulllineitems}
\phantomsection\label{qikify.controllers:qikify.controllers.OLS.OLS.computeStatistics}\pysiglinewithargsret{\bfcode{computeStatistics}}{}{}
\end{fulllineitems}

\index{dw() (qikify.controllers.OLS.OLS method)}

\begin{fulllineitems}
\phantomsection\label{qikify.controllers:qikify.controllers.OLS.OLS.dw}\pysiglinewithargsret{\bfcode{dw}}{}{}
Calculates the Durbin-Waston statistic

\end{fulllineitems}

\index{ll() (qikify.controllers.OLS.OLS method)}

\begin{fulllineitems}
\phantomsection\label{qikify.controllers:qikify.controllers.OLS.OLS.ll}\pysiglinewithargsret{\bfcode{ll}}{}{}
Calculate model log-likelihood and two information criteria

\end{fulllineitems}

\index{omni() (qikify.controllers.OLS.OLS method)}

\begin{fulllineitems}
\phantomsection\label{qikify.controllers:qikify.controllers.OLS.OLS.omni}\pysiglinewithargsret{\bfcode{omni}}{}{}
Omnibus test for normality

\end{fulllineitems}

\index{train() (qikify.controllers.OLS.OLS method)}

\begin{fulllineitems}
\phantomsection\label{qikify.controllers:qikify.controllers.OLS.OLS.train}\pysiglinewithargsret{\bfcode{train}}{\emph{X}, \emph{y}, \emph{useQR=True}, \emph{addConstant=True}}{}
Solve y = Xb.
\begin{quote}\begin{description}
\item[{Parameters }] \leavevmode
\textbf{x} : array, shape (M, N)

\textbf{y} : array, shape (M,)

\textbf{useQR} : boolean
\begin{quote}

Whether or not to use QR decomposition to fit regression line.
\end{quote}

\textbf{addConstant: boolean} :
\begin{quote}

Whether or not to add a constant column to X
\end{quote}

\end{description}\end{quote}

\end{fulllineitems}


\end{fulllineitems}



\subsubsection{\texttt{QFFS} Module}
\label{qikify.controllers:qffs-module}\label{qikify.controllers:module-qikify.controllers.QFFS}\index{qikify.controllers.QFFS (module)}\index{QFFS (class in qikify.controllers.QFFS)}

\begin{fulllineitems}
\phantomsection\label{qikify.controllers:qikify.controllers.QFFS.QFFS}\pysigline{\strong{class }\code{qikify.controllers.QFFS.}\bfcode{QFFS}}
Bases: \code{object}

Qikify feature selection library. Doesn't do much yet; right now only implements
correlation coefficient-based feature selection.
\paragraph{Methods}

\begin{longtable}{ll}
\hline
\endfirsthead

\multicolumn{2}{c}%
{{\bfseries \tablename\ \thetable{} -- continued from previous page}} \\
\hline
\endhead

\hline \multicolumn{2}{|r|}{{Continued on next page}} \\ \hline
\endfoot

\hline
\endlastfoot


{\hyperref[qikify.controllers:qikify.controllers.QFFS.QFFS.computeCorrCoefs]{\code{computeCorrCoefs}}}(X, y)
 & 
Returns the correlation coefficients between X and y,
\\\hline

{\hyperref[qikify.controllers:qikify.controllers.QFFS.QFFS.run]{\code{run}}}(X, y{[}, n\_features, intercept, method{]})
 & 
Do feature selection on the basis of correlation coefficients.
\\\hline
\end{longtable}

\index{computeCorrCoefs() (qikify.controllers.QFFS.QFFS method)}

\begin{fulllineitems}
\phantomsection\label{qikify.controllers:qikify.controllers.QFFS.QFFS.computeCorrCoefs}\pysiglinewithargsret{\bfcode{computeCorrCoefs}}{\emph{X}, \emph{y}}{}
Returns the correlation coefficients between X and y, 
along with the arg-sorted indices of ranked most-correlated X-to-y vars.

\end{fulllineitems}

\index{run() (qikify.controllers.QFFS.QFFS method)}

\begin{fulllineitems}
\phantomsection\label{qikify.controllers:qikify.controllers.QFFS.QFFS.run}\pysiglinewithargsret{\bfcode{run}}{\emph{X}, \emph{y}, \emph{n\_features=10}, \emph{intercept=True}, \emph{method='corrcoef'}}{}
Do feature selection on the basis of correlation coefficients.
\begin{quote}\begin{description}
\item[{Parameters }] \leavevmode
\textbf{X} : numpy array of shape {[}n\_samples,n\_features{]}
\begin{quote}

Training data
\end{quote}

\textbf{y} : numpy array of shape {[}n\_samples{]}
\begin{quote}

Target values
\end{quote}

\textbf{n\_features} : int, optional
\begin{quote}

Number of features to retain
\end{quote}

\textbf{intercept} : bool, optional
\begin{quote}

Whether the first column is an all-constant intercept and 
should be excluded
\end{quote}

\textbf{method} : string, optional
\begin{quote}

Determines the feature selection method to use.
\end{quote}

\item[{Returns }] \leavevmode
\textbf{features} : The X column indices to retain.

\end{description}\end{quote}
\paragraph{Notes}

We typically exclude the first column since it is the intercept
all-constant column.

\end{fulllineitems}


\end{fulllineitems}



\subsubsection{\texttt{SVM} Module}
\label{qikify.controllers:svm-module}\label{qikify.controllers:module-qikify.controllers.SVM}\index{qikify.controllers.SVM (module)}\index{SVM (class in qikify.controllers.SVM)}

\begin{fulllineitems}
\phantomsection\label{qikify.controllers:qikify.controllers.SVM.SVM}\pysigline{\strong{class }\code{qikify.controllers.SVM.}\bfcode{SVM}}
Bases: \code{object}
\paragraph{Methods}

\begin{longtable}{ll}
\hline
\endfirsthead

\multicolumn{2}{c}%
{{\bfseries \tablename\ \thetable{} -- continued from previous page}} \\
\hline
\endhead

\hline \multicolumn{2}{|r|}{{Continued on next page}} \\ \hline
\endfoot

\hline
\endlastfoot


{\hyperref[qikify.controllers:qikify.controllers.SVM.SVM.getTEYL]{\code{getTEYL}}}(gnd, predicted)
 & 

\\\hline

{\hyperref[qikify.controllers:qikify.controllers.SVM.SVM.predict]{\code{predict}}}(X)
 & 

\\\hline

{\hyperref[qikify.controllers:qikify.controllers.SVM.SVM.train]{\code{train}}}(X, gnd{[}, gridSearch{]})
 & 

\\\hline
\end{longtable}

\index{getTEYL() (qikify.controllers.SVM.SVM method)}

\begin{fulllineitems}
\phantomsection\label{qikify.controllers:qikify.controllers.SVM.SVM.getTEYL}\pysiglinewithargsret{\bfcode{getTEYL}}{\emph{gnd}, \emph{predicted}}{}
\end{fulllineitems}

\index{predict() (qikify.controllers.SVM.SVM method)}

\begin{fulllineitems}
\phantomsection\label{qikify.controllers:qikify.controllers.SVM.SVM.predict}\pysiglinewithargsret{\bfcode{predict}}{\emph{X}}{}
\end{fulllineitems}

\index{train() (qikify.controllers.SVM.SVM method)}

\begin{fulllineitems}
\phantomsection\label{qikify.controllers:qikify.controllers.SVM.SVM.train}\pysiglinewithargsret{\bfcode{train}}{\emph{X}, \emph{gnd}, \emph{gridSearch=False}}{}
\end{fulllineitems}


\end{fulllineitems}



\subsubsection{\texttt{identifyOutliers} Module}
\label{qikify.controllers:module-qikify.controllers.identifyOutliers}\label{qikify.controllers:identifyoutliers-module}\index{qikify.controllers.identifyOutliers (module)}\index{identifyOutliers() (in module qikify.controllers.identifyOutliers)}

\begin{fulllineitems}
\phantomsection\label{qikify.controllers:qikify.controllers.identifyOutliers.identifyOutliers}\pysiglinewithargsret{\code{qikify.controllers.identifyOutliers.}\bfcode{identifyOutliers}}{\emph{data}, \emph{k=3}}{}
Compare a dataset against mu +/- k*sigma limits, and
return a boolean vector with False elements denoting outliers.
\begin{quote}\begin{description}
\item[{Parameters }] \leavevmode
\textbf{data} : Contains data stored in a pandas DataFrame or Series.

\end{description}\end{quote}

\end{fulllineitems}

\index{identifyOutliersSpecs() (in module qikify.controllers.identifyOutliers)}

\begin{fulllineitems}
\phantomsection\label{qikify.controllers:qikify.controllers.identifyOutliers.identifyOutliersSpecs}\pysiglinewithargsret{\code{qikify.controllers.identifyOutliers.}\bfcode{identifyOutliersSpecs}}{\emph{data}, \emph{specs}, \emph{ind}, \emph{k=3}}{}
Compare a dataset against expanded spec limits, and
return a boolean vector with False elements denoting outliers.
\begin{quote}\begin{description}
\item[{Parameters }] \leavevmode
\textbf{data} : Contains data stored in a pandas DataFrame or Series.

\end{description}\end{quote}

\end{fulllineitems}



\subsubsection{\texttt{interpolate} Module}
\label{qikify.controllers:module-qikify.controllers.interpolate}\label{qikify.controllers:interpolate-module}\index{qikify.controllers.interpolate (module)}\index{bilinear\_interp() (in module qikify.controllers.interpolate)}

\begin{fulllineitems}
\phantomsection\label{qikify.controllers:qikify.controllers.interpolate.bilinear_interp}\pysiglinewithargsret{\code{qikify.controllers.interpolate.}\bfcode{bilinear\_interp}}{\emph{x}, \emph{y}, \emph{xlim}, \emph{ylim}, \emph{Q}}{}
bilinear interpolation of z over 2d surface \{x,y\}

\end{fulllineitems}

\index{cart2polar() (in module qikify.controllers.interpolate)}

\begin{fulllineitems}
\phantomsection\label{qikify.controllers:qikify.controllers.interpolate.cart2polar}\pysiglinewithargsret{\code{qikify.controllers.interpolate.}\bfcode{cart2polar}}{\emph{x}, \emph{y}}{}
\end{fulllineitems}

\index{cart2polar\_recenter() (in module qikify.controllers.interpolate)}

\begin{fulllineitems}
\phantomsection\label{qikify.controllers:qikify.controllers.interpolate.cart2polar_recenter}\pysiglinewithargsret{\code{qikify.controllers.interpolate.}\bfcode{cart2polar\_recenter}}{\emph{x}, \emph{y}, \emph{xmax}, \emph{ymax}}{}
\end{fulllineitems}

\index{lerp() (in module qikify.controllers.interpolate)}

\begin{fulllineitems}
\phantomsection\label{qikify.controllers:qikify.controllers.interpolate.lerp}\pysiglinewithargsret{\code{qikify.controllers.interpolate.}\bfcode{lerp}}{\emph{x}, \emph{xlim}, \emph{ylim}}{}
linearly interpolate a value of y given ranges for x, y.
\begin{description}
\item[{arguments:}] \leavevmode
x: scalar
xlim: array with xmin, xmax
ylim: array with ymin, ymax

\end{description}

\end{fulllineitems}

\index{polar2cart() (in module qikify.controllers.interpolate)}

\begin{fulllineitems}
\phantomsection\label{qikify.controllers:qikify.controllers.interpolate.polar2cart}\pysiglinewithargsret{\code{qikify.controllers.interpolate.}\bfcode{polar2cart}}{\emph{r}, \emph{theta}}{}
\end{fulllineitems}

\index{polar2cart\_recenter() (in module qikify.controllers.interpolate)}

\begin{fulllineitems}
\phantomsection\label{qikify.controllers:qikify.controllers.interpolate.polar2cart_recenter}\pysiglinewithargsret{\code{qikify.controllers.interpolate.}\bfcode{polar2cart\_recenter}}{\emph{r}, \emph{theta}, \emph{xmax}, \emph{ymax}}{}
\end{fulllineitems}



\subsubsection{\texttt{slicesample} Module}
\label{qikify.controllers:module-qikify.controllers.slicesample}\label{qikify.controllers:slicesample-module}\index{qikify.controllers.slicesample (module)}\index{inside() (in module qikify.controllers.slicesample)}

\begin{fulllineitems}
\phantomsection\label{qikify.controllers:qikify.controllers.slicesample.inside}\pysiglinewithargsret{\code{qikify.controllers.slicesample.}\bfcode{inside}}{\emph{x}, \emph{th}, \emph{pdf}}{}
\end{fulllineitems}

\index{logpdf() (in module qikify.controllers.slicesample)}

\begin{fulllineitems}
\phantomsection\label{qikify.controllers:qikify.controllers.slicesample.logpdf}\pysiglinewithargsret{\code{qikify.controllers.slicesample.}\bfcode{logpdf}}{\emph{x}, \emph{pdf}}{}
\end{fulllineitems}

\index{outside() (in module qikify.controllers.slicesample)}

\begin{fulllineitems}
\phantomsection\label{qikify.controllers:qikify.controllers.slicesample.outside}\pysiglinewithargsret{\code{qikify.controllers.slicesample.}\bfcode{outside}}{\emph{x}, \emph{th}, \emph{pdf}}{}
\end{fulllineitems}

\index{slicesample() (in module qikify.controllers.slicesample)}

\begin{fulllineitems}
\phantomsection\label{qikify.controllers:qikify.controllers.slicesample.slicesample}\pysiglinewithargsret{\code{qikify.controllers.slicesample.}\bfcode{slicesample}}{\emph{x0}, \emph{nsamples}, \emph{pdf}, \emph{width=10}, \emph{maxiter=200}}{}
Loosely based on slicesample() from MATLAB.

\end{fulllineitems}



\subsection{models Package}
\label{qikify.models::doc}\label{qikify.models:models-package}

\subsubsection{\texttt{chip} Module}
\label{qikify.models:chip-module}\label{qikify.models:module-qikify.models.chip}\index{qikify.models.chip (module)}\index{Chip (class in qikify.models.chip)}

\begin{fulllineitems}
\phantomsection\label{qikify.models:qikify.models.chip.Chip}\pysiglinewithargsret{\strong{class }\code{qikify.models.chip.}\bfcode{Chip}}{\emph{chip\_dict}, \emph{LCT\_prefix='`}}{}
Bases: \code{object}

This class encapsulates chip-level data.

\end{fulllineitems}



\subsubsection{\texttt{dataset} Module}
\label{qikify.models:module-qikify.models.dataset}\label{qikify.models:dataset-module}\index{qikify.models.dataset (module)}
\begin{notice}{warning}{Warning:}
Deprecated in version 0.2.
\end{notice}
\index{Dataset (class in qikify.models.dataset)}

\begin{fulllineitems}
\phantomsection\label{qikify.models:qikify.models.dataset.Dataset}\pysiglinewithargsret{\strong{class }\code{qikify.models.dataset.}\bfcode{Dataset}}{\emph{filename=None}, \emph{files=None}, \emph{dataset=None}}{}
Bases: {\hyperref[qikify.models:qikify.models.dotdict.dotdict]{\code{qikify.models.dotdict.dotdict}}}

This class is the fundamental data structure of the Qikify framework.
\paragraph{Methods}

\begin{longtable}{ll}
\hline
\endfirsthead

\multicolumn{2}{c}%
{{\bfseries \tablename\ \thetable{} -- continued from previous page}} \\
\hline
\endhead

\hline \multicolumn{2}{|r|}{{Continued on next page}} \\ \hline
\endfoot

\hline
\endlastfoot


\code{clear}(() -\textgreater{} None.  Remove all items from D.)
 & 

\\\hline

\code{copy}(() -\textgreater{} a shallow copy of D)
 & 

\\\hline

\code{fromkeys}(...)
 & 
v defaults to None.
\\\hline

\code{get}((k{[},d{]}) -\textgreater{} D{[}k{]} if k in D, ...)
 & 

\\\hline

\code{has\_key}((k) -\textgreater{} True if D has a key k, else False)
 & 

\\\hline

\code{items}(() -\textgreater{} list of D's (key, value) pairs, ...)
 & 

\\\hline

\code{iteritems}(() -\textgreater{} an iterator over the (key, ...)
 & 

\\\hline

\code{iterkeys}(() -\textgreater{} an iterator over the keys of D)
 & 

\\\hline

\code{itervalues}(...)
 & 

\\\hline

\code{keys}(() -\textgreater{} list of D's keys)
 & 

\\\hline

\code{pop}((k{[},d{]}) -\textgreater{} v, ...)
 & 
If key is not found, d is returned if given, otherwise KeyError is raised
\\\hline

\code{popitem}(() -\textgreater{} (k, v), ...)
 & 
2-tuple; but raise KeyError if D is empty.
\\\hline

\code{setdefault}((k{[},d{]}) -\textgreater{} D.get(k,d), ...)
 & 

\\\hline

\code{update}((E, ...)
 & 
If E has a .keys() method, does:     for k in E: D{[}k{]} = E{[}k{]}
\\\hline

\code{values}(() -\textgreater{} list of D's values)
 & 

\\\hline

\code{viewitems}(...)
 & 

\\\hline

\code{viewkeys}(...)
 & 

\\\hline

\code{viewvalues}(...)
 & 

\\\hline
\end{longtable}


\end{fulllineitems}



\subsubsection{\texttt{dotdict} Module}
\label{qikify.models:dotdict-module}\label{qikify.models:module-qikify.models.dotdict}\index{qikify.models.dotdict (module)}\index{dotdict (class in qikify.models.dotdict)}

\begin{fulllineitems}
\phantomsection\label{qikify.models:qikify.models.dotdict.dotdict}\pysigline{\strong{class }\code{qikify.models.dotdict.}\bfcode{dotdict}}
Bases: \code{dict}

We use dotdict to replace standard Python dictionaries. This 
is simply for the convenience of having dict.property access,
instead of the messier dict{[}'property'{]} style.
\paragraph{Methods}

\begin{longtable}{ll}
\hline
\endfirsthead

\multicolumn{2}{c}%
{{\bfseries \tablename\ \thetable{} -- continued from previous page}} \\
\hline
\endhead

\hline \multicolumn{2}{|r|}{{Continued on next page}} \\ \hline
\endfoot

\hline
\endlastfoot


\code{clear}(() -\textgreater{} None.  Remove all items from D.)
 & 

\\\hline

\code{copy}(() -\textgreater{} a shallow copy of D)
 & 

\\\hline

\code{fromkeys}(...)
 & 
v defaults to None.
\\\hline

\code{get}((k{[},d{]}) -\textgreater{} D{[}k{]} if k in D, ...)
 & 

\\\hline

\code{has\_key}((k) -\textgreater{} True if D has a key k, else False)
 & 

\\\hline

\code{items}(() -\textgreater{} list of D's (key, value) pairs, ...)
 & 

\\\hline

\code{iteritems}(() -\textgreater{} an iterator over the (key, ...)
 & 

\\\hline

\code{iterkeys}(() -\textgreater{} an iterator over the keys of D)
 & 

\\\hline

\code{itervalues}(...)
 & 

\\\hline

\code{keys}(() -\textgreater{} list of D's keys)
 & 

\\\hline

\code{pop}((k{[},d{]}) -\textgreater{} v, ...)
 & 
If key is not found, d is returned if given, otherwise KeyError is raised
\\\hline

\code{popitem}(() -\textgreater{} (k, v), ...)
 & 
2-tuple; but raise KeyError if D is empty.
\\\hline

\code{setdefault}((k{[},d{]}) -\textgreater{} D.get(k,d), ...)
 & 

\\\hline

\code{update}((E, ...)
 & 
If E has a .keys() method, does:     for k in E: D{[}k{]} = E{[}k{]}
\\\hline

\code{values}(() -\textgreater{} list of D's values)
 & 

\\\hline

\code{viewitems}(...)
 & 

\\\hline

\code{viewkeys}(...)
 & 

\\\hline

\code{viewvalues}(...)
 & 

\\\hline
\end{longtable}


\end{fulllineitems}

\index{mdotmap (class in qikify.models.dotdict)}

\begin{fulllineitems}
\phantomsection\label{qikify.models:qikify.models.dotdict.mdotmap}\pysiglinewithargsret{\strong{class }\code{qikify.models.dotdict.}\bfcode{mdotmap}}{\emph{*args}, \emph{**kwargs}}{}
Bases: \code{\_abcoll.MutableMapping}

We use mdotmap to replace standard Python dictionaries. This 
is simply for the convenience of having mdotmap.attr access,
instead of the dict{[}attr{]} style.

** NOT YET WORKING **
\paragraph{Methods}

\begin{longtable}{ll}
\hline
\endfirsthead

\multicolumn{2}{c}%
{{\bfseries \tablename\ \thetable{} -- continued from previous page}} \\
\hline
\endhead

\hline \multicolumn{2}{|r|}{{Continued on next page}} \\ \hline
\endfoot

\hline
\endlastfoot


\code{clear}()
 & 

\\\hline

\code{get}(key{[}, default{]})
 & 

\\\hline

\code{items}()
 & 

\\\hline

\code{iteritems}()
 & 

\\\hline

\code{iterkeys}()
 & 

\\\hline

\code{itervalues}()
 & 

\\\hline

\code{keys}()
 & 

\\\hline

\code{pop}(key{[}, default{]})
 & 

\\\hline

\code{popitem}()
 & 

\\\hline

\code{setdefault}(key{[}, default{]})
 & 

\\\hline

\code{update}(*args, **kwds)
 & 

\\\hline

\code{values}()
 & 

\\\hline
\end{longtable}


\end{fulllineitems}



\subsubsection{\texttt{helpers} Module}
\label{qikify.models:helpers-module}\label{qikify.models:module-qikify.models.helpers}\index{qikify.models.helpers (module)}\index{gz\_csv\_read() (in module qikify.models.helpers)}

\begin{fulllineitems}
\phantomsection\label{qikify.models:qikify.models.helpers.gz_csv_read}\pysiglinewithargsret{\code{qikify.models.helpers.}\bfcode{gz\_csv\_read}}{\emph{file\_path}, \emph{pandasDF=False}}{}
\end{fulllineitems}

\index{gz\_csv\_write() (in module qikify.models.helpers)}

\begin{fulllineitems}
\phantomsection\label{qikify.models:qikify.models.helpers.gz_csv_write}\pysiglinewithargsret{\code{qikify.models.helpers.}\bfcode{gz\_csv\_write}}{\emph{file\_path}, \emph{data}}{}
\end{fulllineitems}



\subsubsection{\texttt{specs} Module}
\label{qikify.models:module-qikify.models.specs}\label{qikify.models:specs-module}\index{qikify.models.specs (module)}\index{Specs (class in qikify.models.specs)}

\begin{fulllineitems}
\phantomsection\label{qikify.models:qikify.models.specs.Specs}\pysiglinewithargsret{\strong{class }\code{qikify.models.specs.}\bfcode{Specs}}{\emph{filename=None}, \emph{specs=None}}{}
Bases: \code{object}
\paragraph{Methods}

\begin{longtable}{ll}
\hline
\endfirsthead

\multicolumn{2}{c}%
{{\bfseries \tablename\ \thetable{} -- continued from previous page}} \\
\hline
\endhead

\hline \multicolumn{2}{|r|}{{Continued on next page}} \\ \hline
\endfoot

\hline
\endlastfoot


{\hyperref[qikify.models:qikify.models.specs.Specs.computePassFail]{\code{computePassFail}}}(data)
 & 
Compare a pandas Series or DataFrame structure to specification limits defined by
\\\hline

{\hyperref[qikify.models:qikify.models.specs.Specs.genCriticalRegion]{\code{genCriticalRegion}}}(k\_i, k\_o)
 & 
Takes specification boundary and generates two boundaries to define `critical' device  region.
\\\hline
\end{longtable}

\index{computePassFail() (qikify.models.specs.Specs method)}

\begin{fulllineitems}
\phantomsection\label{qikify.models:qikify.models.specs.Specs.computePassFail}\pysiglinewithargsret{\bfcode{computePassFail}}{\emph{data}}{}
Compare a pandas Series or DataFrame structure to specification limits defined by
this spec class instance.
\begin{quote}\begin{description}
\item[{Parameters }] \leavevmode
\textbf{data} : Contains data stored in Series or DataFrame.

\end{description}\end{quote}

\end{fulllineitems}

\index{genCriticalRegion() (qikify.models.specs.Specs method)}

\begin{fulllineitems}
\phantomsection\label{qikify.models:qikify.models.specs.Specs.genCriticalRegion}\pysiglinewithargsret{\bfcode{genCriticalRegion}}{\emph{k\_i}, \emph{k\_o}}{}
Takes specification boundary and generates two boundaries to define `critical' device 
region.
\begin{quote}\begin{description}
\item[{Parameters }] \leavevmode
\textbf{k\_i} : Inner critical region multiplier.

\textbf{k\_u} : Outer critical region multiplier.

\end{description}\end{quote}

\end{fulllineitems}


\end{fulllineitems}



\subsection{recipes Package}
\label{qikify.recipes::doc}\label{qikify.recipes:recipes-package}

\subsubsection{\texttt{atesim} Module}
\label{qikify.recipes:atesim-module}\label{qikify.recipes:module-qikify.recipes.atesim}\index{qikify.recipes.atesim (module)}\index{ATESimulator (class in qikify.recipes.atesim)}

\begin{fulllineitems}
\phantomsection\label{qikify.recipes:qikify.recipes.atesim.ATESimulator}\pysiglinewithargsret{\strong{class }\code{qikify.recipes.atesim.}\bfcode{ATESimulator}}{\emph{data\_src='filesystem'}}{}
Bases: \code{object}
\paragraph{Methods}

\begin{longtable}{ll}
\hline
\endfirsthead

\multicolumn{2}{c}%
{{\bfseries \tablename\ \thetable{} -- continued from previous page}} \\
\hline
\endhead

\hline \multicolumn{2}{|r|}{{Continued on next page}} \\ \hline
\endfoot

\hline
\endlastfoot


{\hyperref[qikify.recipes:qikify.recipes.atesim.ATESimulator.run]{\code{run}}}({[}port{]})
 & 
This function runs the ATE simulator using CSV files in the current directory.
\\\hline
\end{longtable}

\index{run() (qikify.recipes.atesim.ATESimulator method)}

\begin{fulllineitems}
\phantomsection\label{qikify.recipes:qikify.recipes.atesim.ATESimulator.run}\pysiglinewithargsret{\bfcode{run}}{\emph{port=5570}}{}
This function runs the ATE simulator using CSV files in the current directory.
Currently, we only support loading .csv or .csv.gz files.

\end{fulllineitems}


\end{fulllineitems}

\index{ChipDataIterator (class in qikify.recipes.atesim)}

\begin{fulllineitems}
\phantomsection\label{qikify.recipes:qikify.recipes.atesim.ChipDataIterator}\pysiglinewithargsret{\strong{class }\code{qikify.recipes.atesim.}\bfcode{ChipDataIterator}}{\emph{data\_dir}}{}
Bases: \code{object}
\paragraph{Methods}

\begin{longtable}{ll}
\hline
\endfirsthead

\multicolumn{2}{c}%
{{\bfseries \tablename\ \thetable{} -- continued from previous page}} \\
\hline
\endhead

\hline \multicolumn{2}{|r|}{{Continued on next page}} \\ \hline
\endfoot

\hline
\endlastfoot


{\hyperref[qikify.recipes:qikify.recipes.atesim.ChipDataIterator.next]{\code{next}}}()
 & 
The call to self.chip\_iter.next() will raise StopIteration when done,
\\\hline
\end{longtable}

\index{next() (qikify.recipes.atesim.ChipDataIterator method)}

\begin{fulllineitems}
\phantomsection\label{qikify.recipes:qikify.recipes.atesim.ChipDataIterator.next}\pysiglinewithargsret{\bfcode{next}}{}{}
The call to self.chip\_iter.next() will raise StopIteration when done, 
propagating through to the caller of ChipDataIterator().next().

\end{fulllineitems}


\end{fulllineitems}



\subsubsection{\texttt{basic\_ML\_testing} Module}
\label{qikify.recipes:module-qikify.recipes.basic_ML_testing}\label{qikify.recipes:basic-ml-testing-module}\index{qikify.recipes.basic\_ML\_testing (module)}\index{BasicMLTesting (class in qikify.recipes.basic\_ML\_testing)}

\begin{fulllineitems}
\phantomsection\label{qikify.recipes:qikify.recipes.basic_ML_testing.BasicMLTesting}\pysigline{\strong{class }\code{qikify.recipes.basic\_ML\_testing.}\bfcode{BasicMLTesting}}
Bases: \code{object}
\paragraph{Methods}

\begin{longtable}{ll}
\hline
\endfirsthead

\multicolumn{2}{c}%
{{\bfseries \tablename\ \thetable{} -- continued from previous page}} \\
\hline
\endhead

\hline \multicolumn{2}{|r|}{{Continued on next page}} \\ \hline
\endfoot

\hline
\endlastfoot


{\hyperref[qikify.recipes:qikify.recipes.basic_ML_testing.BasicMLTesting.run]{\code{run}}}({[}port{]})
 & 

\\\hline
\end{longtable}

\index{run() (qikify.recipes.basic\_ML\_testing.BasicMLTesting method)}

\begin{fulllineitems}
\phantomsection\label{qikify.recipes:qikify.recipes.basic_ML_testing.BasicMLTesting.run}\pysiglinewithargsret{\bfcode{run}}{\emph{port=5570}}{}
\end{fulllineitems}


\end{fulllineitems}



\subsubsection{\texttt{two\_tier\_test} Module}
\label{qikify.recipes:module-qikify.recipes.two_tier_test}\label{qikify.recipes:two-tier-test-module}\index{qikify.recipes.two\_tier\_test (module)}\index{TwoTierTest (class in qikify.recipes.two\_tier\_test)}

\begin{fulllineitems}
\phantomsection\label{qikify.recipes:qikify.recipes.two_tier_test.TwoTierTest}\pysigline{\strong{class }\code{qikify.recipes.two\_tier\_test.}\bfcode{TwoTierTest}}
Bases: \code{object}
\paragraph{Methods}

\begin{longtable}{ll}
\hline
\endfirsthead

\multicolumn{2}{c}%
{{\bfseries \tablename\ \thetable{} -- continued from previous page}} \\
\hline
\endhead

\hline \multicolumn{2}{|r|}{{Continued on next page}} \\ \hline
\endfoot

\hline
\endlastfoot


{\hyperref[qikify.recipes:qikify.recipes.two_tier_test.TwoTierTest.run]{\code{run}}}({[}port{]})
 & 

\\\hline
\end{longtable}

\index{run() (qikify.recipes.two\_tier\_test.TwoTierTest method)}

\begin{fulllineitems}
\phantomsection\label{qikify.recipes:qikify.recipes.two_tier_test.TwoTierTest.run}\pysiglinewithargsret{\bfcode{run}}{\emph{port=5570}}{}
\end{fulllineitems}


\end{fulllineitems}



\subsection{views Package}
\label{qikify.views:views-package}\label{qikify.views::doc}

\subsubsection{\texttt{charts} Module}
\label{qikify.views:charts-module}\label{qikify.views:module-qikify.views.charts}\index{qikify.views.charts (module)}\index{coef\_path() (in module qikify.views.charts)}

\begin{fulllineitems}
\phantomsection\label{qikify.views:qikify.views.charts.coef_path}\pysiglinewithargsret{\code{qikify.views.charts.}\bfcode{coef\_path}}{\emph{coefs}}{}
Plot the coefficient paths generated by elastic net / lasso.

\end{fulllineitems}

\index{histogram() (in module qikify.views.charts)}

\begin{fulllineitems}
\phantomsection\label{qikify.views:qikify.views.charts.histogram}\pysiglinewithargsret{\code{qikify.views.charts.}\bfcode{histogram}}{\emph{sData}, \emph{bData}, \emph{i}, \emph{filename=None}}{}
\end{fulllineitems}

\index{laplacianScores() (in module qikify.views.charts)}

\begin{fulllineitems}
\phantomsection\label{qikify.views:qikify.views.charts.laplacianScores}\pysiglinewithargsret{\code{qikify.views.charts.}\bfcode{laplacianScores}}{\emph{filename}, \emph{Scores}, \emph{Ranking}}{}
\end{fulllineitems}

\index{pairs() (in module qikify.views.charts)}

\begin{fulllineitems}
\phantomsection\label{qikify.views:qikify.views.charts.pairs}\pysiglinewithargsret{\code{qikify.views.charts.}\bfcode{pairs}}{\emph{data}, \emph{labels=None}, \emph{filename=None}}{}
Generates something similar to R pairs()

\end{fulllineitems}

\index{percentFormatter() (in module qikify.views.charts)}

\begin{fulllineitems}
\phantomsection\label{qikify.views:qikify.views.charts.percentFormatter}\pysiglinewithargsret{\code{qikify.views.charts.}\bfcode{percentFormatter}}{\emph{x}, \emph{pos=0}}{}
\end{fulllineitems}

\index{qq() (in module qikify.views.charts)}

\begin{fulllineitems}
\phantomsection\label{qikify.views:qikify.views.charts.qq}\pysiglinewithargsret{\code{qikify.views.charts.}\bfcode{qq}}{\emph{x}, \emph{filename=None}}{}
\end{fulllineitems}

\index{syntheticAndReal() (in module qikify.views.charts)}

\begin{fulllineitems}
\phantomsection\label{qikify.views:qikify.views.charts.syntheticAndReal}\pysiglinewithargsret{\code{qikify.views.charts.}\bfcode{syntheticAndReal}}{\emph{sData}, \emph{bData}, \emph{d1}, \emph{d2}, \emph{filename}}{}
\end{fulllineitems}

\index{te\_and\_yl() (in module qikify.views.charts)}

\begin{fulllineitems}
\phantomsection\label{qikify.views:qikify.views.charts.te_and_yl}\pysiglinewithargsret{\code{qikify.views.charts.}\bfcode{te\_and\_yl}}{\emph{error}, \emph{errorSyn}, \emph{filename}, \emph{description}}{}
\end{fulllineitems}

\index{wafermap() (in module qikify.views.charts)}

\begin{fulllineitems}
\phantomsection\label{qikify.views:qikify.views.charts.wafermap}\pysiglinewithargsret{\code{qikify.views.charts.}\bfcode{wafermap}}{\emph{x}, \emph{y}, \emph{val}, \emph{filename=None}}{}
Plots a heatmap of argument val over wafer coordinates.

\end{fulllineitems}

\index{yp\_vs\_y() (in module qikify.views.charts)}

\begin{fulllineitems}
\phantomsection\label{qikify.views:qikify.views.charts.yp_vs_y}\pysiglinewithargsret{\code{qikify.views.charts.}\bfcode{yp\_vs\_y}}{\emph{yp}, \emph{y}, \emph{filename=None}}{}
This method plots y predicted vs. y actual on a 45-degree chart.

\end{fulllineitems}



\chapter{Indices and tables}
\label{index:indices-and-tables}\begin{itemize}
\item {} 
\emph{genindex}

\item {} 
\emph{modindex}

\item {} 
\emph{search}

\end{itemize}

\begin{thebibliography}{R1}
\bibitem[R1]{R1}{\phantomsection\label{qikify.controllers:r1} 
He, X. and Cai, D. and Niyogi, P., ``Laplacian Score for Feature Selection'', NIPS 2005.
}
\end{thebibliography}


\renewcommand{\indexname}{Python Module Index}
\begin{theindex}
\def\bigletter#1{{\Large\sffamily#1}\nopagebreak\vspace{1mm}}
\bigletter{q}
\item {\texttt{qikify.\_\_init\_\_}}, \pageref{qikify:module-qikify.__init__}
\item {\texttt{qikify.controllers.identifyOutliers}}, \pageref{qikify.controllers:module-qikify.controllers.identifyOutliers}
\item {\texttt{qikify.controllers.interpolate}}, \pageref{qikify.controllers:module-qikify.controllers.interpolate}
\item {\texttt{qikify.controllers.KDE}}, \pageref{qikify.controllers:module-qikify.controllers.KDE}
\item {\texttt{qikify.controllers.LSFS}}, \pageref{qikify.controllers:module-qikify.controllers.LSFS}
\item {\texttt{qikify.controllers.OLS}}, \pageref{qikify.controllers:module-qikify.controllers.OLS}
\item {\texttt{qikify.controllers.QFFS}}, \pageref{qikify.controllers:module-qikify.controllers.QFFS}
\item {\texttt{qikify.controllers.slicesample}}, \pageref{qikify.controllers:module-qikify.controllers.slicesample}
\item {\texttt{qikify.controllers.SVM}}, \pageref{qikify.controllers:module-qikify.controllers.SVM}
\item {\texttt{qikify.helpers}}, \pageref{qikify:module-qikify.helpers}
\item {\texttt{qikify.models.chip}}, \pageref{qikify.models:module-qikify.models.chip}
\item {\texttt{qikify.models.dataset}}, \pageref{qikify.models:module-qikify.models.dataset}
\item {\texttt{qikify.models.dotdict}}, \pageref{qikify.models:module-qikify.models.dotdict}
\item {\texttt{qikify.models.helpers}}, \pageref{qikify.models:module-qikify.models.helpers}
\item {\texttt{qikify.models.specs}}, \pageref{qikify.models:module-qikify.models.specs}
\item {\texttt{qikify.recipes.atesim}}, \pageref{qikify.recipes:module-qikify.recipes.atesim}
\item {\texttt{qikify.recipes.basic\_ML\_testing}}, \pageref{qikify.recipes:module-qikify.recipes.basic_ML_testing}
\item {\texttt{qikify.recipes.two\_tier\_test}}, \pageref{qikify.recipes:module-qikify.recipes.two_tier_test}
\item {\texttt{qikify.term\_helpers}}, \pageref{qikify:module-qikify.term_helpers}
\item {\texttt{qikify.views.charts}}, \pageref{qikify.views:module-qikify.views.charts}
\end{theindex}
\renewcommand{\indexname}{Python Module Index}
\begin{theindex}
\def\bigletter#1{{\Large\sffamily#1}\nopagebreak\vspace{1mm}}
\bigletter{q}
\item {\texttt{qikify.\_\_init\_\_}}, \pageref{qikify:module-qikify.__init__}
\item {\texttt{qikify.controllers.identifyOutliers}}, \pageref{qikify.controllers:module-qikify.controllers.identifyOutliers}
\item {\texttt{qikify.controllers.interpolate}}, \pageref{qikify.controllers:module-qikify.controllers.interpolate}
\item {\texttt{qikify.controllers.KDE}}, \pageref{qikify.controllers:module-qikify.controllers.KDE}
\item {\texttt{qikify.controllers.LSFS}}, \pageref{qikify.controllers:module-qikify.controllers.LSFS}
\item {\texttt{qikify.controllers.OLS}}, \pageref{qikify.controllers:module-qikify.controllers.OLS}
\item {\texttt{qikify.controllers.QFFS}}, \pageref{qikify.controllers:module-qikify.controllers.QFFS}
\item {\texttt{qikify.controllers.slicesample}}, \pageref{qikify.controllers:module-qikify.controllers.slicesample}
\item {\texttt{qikify.controllers.SVM}}, \pageref{qikify.controllers:module-qikify.controllers.SVM}
\item {\texttt{qikify.helpers}}, \pageref{qikify:module-qikify.helpers}
\item {\texttt{qikify.models.chip}}, \pageref{qikify.models:module-qikify.models.chip}
\item {\texttt{qikify.models.dataset}}, \pageref{qikify.models:module-qikify.models.dataset}
\item {\texttt{qikify.models.dotdict}}, \pageref{qikify.models:module-qikify.models.dotdict}
\item {\texttt{qikify.models.helpers}}, \pageref{qikify.models:module-qikify.models.helpers}
\item {\texttt{qikify.models.specs}}, \pageref{qikify.models:module-qikify.models.specs}
\item {\texttt{qikify.recipes.atesim}}, \pageref{qikify.recipes:module-qikify.recipes.atesim}
\item {\texttt{qikify.recipes.basic\_ML\_testing}}, \pageref{qikify.recipes:module-qikify.recipes.basic_ML_testing}
\item {\texttt{qikify.recipes.two\_tier\_test}}, \pageref{qikify.recipes:module-qikify.recipes.two_tier_test}
\item {\texttt{qikify.term\_helpers}}, \pageref{qikify:module-qikify.term_helpers}
\item {\texttt{qikify.views.charts}}, \pageref{qikify.views:module-qikify.views.charts}
\end{theindex}

\renewcommand{\indexname}{Index}
\printindex
\end{document}
